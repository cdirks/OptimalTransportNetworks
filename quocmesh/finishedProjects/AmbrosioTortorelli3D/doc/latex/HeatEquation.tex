%\input{Ambro.tex}
\section{3D Heat Equation Solution with Quocmesh}
The heat equation is an important partial differential equation
which describes the variation of temperature in a given region
over time. In the special case of heat propagation in an isotropic
and homogeneous medium in the 3-dimensional space, this equation
is
\begin{eqnarray}
\partial_t u- \Delta u = 0
\end{eqnarray}
The weak form of above heat equation is
\begin{eqnarray}
 (\partial_t u, \phi) - (\Delta u, \phi) = 0
\label{eq:weakForm_1}
\end{eqnarray}
for all $\phi$. Here $(u,v)=\int_\Omega u(x)\cdot v(x)\;dx$.
According to the derivative rule
\begin{eqnarray*}
(\Delta u, \phi)
=\int_\Omega \nabla\cdot\nabla u\cdot \phi\;dx =
-\int_\Omega\nabla u\cdot\nabla \phi\;dx+
\int_{\partial\Omega}\nabla u \cdot\nu\phi\;dx
\end{eqnarray*}
Because $\phi(x)=0 $ on $\partial\Omega$, so
Eq.(\ref{eq:weakForm_1}) can be rewritten as
\begin{eqnarray}
 (\partial_t u, \phi) +(\nabla u, \nabla \phi)=0
\end{eqnarray}
For each $j=1...N$
\begin{eqnarray}
 \sum_i\partial_t u_i(\phi_i,\phi_j) + \sum_i u_i(
 \nabla\phi_i,\nabla\phi_j)= 0
\end{eqnarray}
Organize the equation system in the way of matrix:
\begin{eqnarray}
M \overline{\partial_t U} + L \overline{U} = 0
\end{eqnarray}
where $M_{i,j}= (\phi_i,\phi_j)$ and $L_{i,j} =
(\nabla\phi_i,\nabla\phi_j)$. For the discretization of time,
 we use implicit backward scheme.
\begin{eqnarray*}
M \frac{\overline{U}^{k+1}-\overline{U}^{k}}{\tau} + L
\overline{U}^{k+1} = 0 \ \Rightarrow \  \overline{U}^{k+1} =
(M+\tau L)^{-1} M \overline{U}^k
\end{eqnarray*}
